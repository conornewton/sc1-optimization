\PassOptionsToPackage{unicode=true}{hyperref} % options for packages loaded elsewhere
\PassOptionsToPackage{hyphens}{url}
%
\documentclass[12pt,]{article}
\usepackage{lmodern}
\usepackage{amssymb,amsmath}
\usepackage{ifxetex,ifluatex}
\usepackage{fixltx2e} % provides \textsubscript
\ifnum 0\ifxetex 1\fi\ifluatex 1\fi=0 % if pdftex
  \usepackage[T1]{fontenc}
  \usepackage[utf8]{inputenc}
  \usepackage{textcomp} % provides euro and other symbols
\else % if luatex or xelatex
  \usepackage{unicode-math}
  \defaultfontfeatures{Ligatures=TeX,Scale=MatchLowercase}
\fi
% use upquote if available, for straight quotes in verbatim environments
\IfFileExists{upquote.sty}{\usepackage{upquote}}{}
% use microtype if available
\IfFileExists{microtype.sty}{%
\usepackage[]{microtype}
\UseMicrotypeSet[protrusion]{basicmath} % disable protrusion for tt fonts
}{}
\IfFileExists{parskip.sty}{%
\usepackage{parskip}
}{% else
\setlength{\parindent}{0pt}
\setlength{\parskip}{6pt plus 2pt minus 1pt}
}
\usepackage{hyperref}
\hypersetup{
            pdftitle={Statistical Computing - Assessed Coursework 1},
            pdfauthor={Conor Newton},
            pdfborder={0 0 0},
            breaklinks=true}
\urlstyle{same}  % don't use monospace font for urls
\usepackage[margin=1in]{geometry}
\usepackage{color}
\usepackage{fancyvrb}
\newcommand{\VerbBar}{|}
\newcommand{\VERB}{\Verb[commandchars=\\\{\}]}
\DefineVerbatimEnvironment{Highlighting}{Verbatim}{commandchars=\\\{\}}
% Add ',fontsize=\small' for more characters per line
\usepackage{framed}
\definecolor{shadecolor}{RGB}{248,248,248}
\newenvironment{Shaded}{\begin{snugshade}}{\end{snugshade}}
\newcommand{\AlertTok}[1]{\textcolor[rgb]{0.94,0.16,0.16}{#1}}
\newcommand{\AnnotationTok}[1]{\textcolor[rgb]{0.56,0.35,0.01}{\textbf{\textit{#1}}}}
\newcommand{\AttributeTok}[1]{\textcolor[rgb]{0.77,0.63,0.00}{#1}}
\newcommand{\BaseNTok}[1]{\textcolor[rgb]{0.00,0.00,0.81}{#1}}
\newcommand{\BuiltInTok}[1]{#1}
\newcommand{\CharTok}[1]{\textcolor[rgb]{0.31,0.60,0.02}{#1}}
\newcommand{\CommentTok}[1]{\textcolor[rgb]{0.56,0.35,0.01}{\textit{#1}}}
\newcommand{\CommentVarTok}[1]{\textcolor[rgb]{0.56,0.35,0.01}{\textbf{\textit{#1}}}}
\newcommand{\ConstantTok}[1]{\textcolor[rgb]{0.00,0.00,0.00}{#1}}
\newcommand{\ControlFlowTok}[1]{\textcolor[rgb]{0.13,0.29,0.53}{\textbf{#1}}}
\newcommand{\DataTypeTok}[1]{\textcolor[rgb]{0.13,0.29,0.53}{#1}}
\newcommand{\DecValTok}[1]{\textcolor[rgb]{0.00,0.00,0.81}{#1}}
\newcommand{\DocumentationTok}[1]{\textcolor[rgb]{0.56,0.35,0.01}{\textbf{\textit{#1}}}}
\newcommand{\ErrorTok}[1]{\textcolor[rgb]{0.64,0.00,0.00}{\textbf{#1}}}
\newcommand{\ExtensionTok}[1]{#1}
\newcommand{\FloatTok}[1]{\textcolor[rgb]{0.00,0.00,0.81}{#1}}
\newcommand{\FunctionTok}[1]{\textcolor[rgb]{0.00,0.00,0.00}{#1}}
\newcommand{\ImportTok}[1]{#1}
\newcommand{\InformationTok}[1]{\textcolor[rgb]{0.56,0.35,0.01}{\textbf{\textit{#1}}}}
\newcommand{\KeywordTok}[1]{\textcolor[rgb]{0.13,0.29,0.53}{\textbf{#1}}}
\newcommand{\NormalTok}[1]{#1}
\newcommand{\OperatorTok}[1]{\textcolor[rgb]{0.81,0.36,0.00}{\textbf{#1}}}
\newcommand{\OtherTok}[1]{\textcolor[rgb]{0.56,0.35,0.01}{#1}}
\newcommand{\PreprocessorTok}[1]{\textcolor[rgb]{0.56,0.35,0.01}{\textit{#1}}}
\newcommand{\RegionMarkerTok}[1]{#1}
\newcommand{\SpecialCharTok}[1]{\textcolor[rgb]{0.00,0.00,0.00}{#1}}
\newcommand{\SpecialStringTok}[1]{\textcolor[rgb]{0.31,0.60,0.02}{#1}}
\newcommand{\StringTok}[1]{\textcolor[rgb]{0.31,0.60,0.02}{#1}}
\newcommand{\VariableTok}[1]{\textcolor[rgb]{0.00,0.00,0.00}{#1}}
\newcommand{\VerbatimStringTok}[1]{\textcolor[rgb]{0.31,0.60,0.02}{#1}}
\newcommand{\WarningTok}[1]{\textcolor[rgb]{0.56,0.35,0.01}{\textbf{\textit{#1}}}}
\usepackage{graphicx,grffile}
\makeatletter
\def\maxwidth{\ifdim\Gin@nat@width>\linewidth\linewidth\else\Gin@nat@width\fi}
\def\maxheight{\ifdim\Gin@nat@height>\textheight\textheight\else\Gin@nat@height\fi}
\makeatother
% Scale images if necessary, so that they will not overflow the page
% margins by default, and it is still possible to overwrite the defaults
% using explicit options in \includegraphics[width, height, ...]{}
\setkeys{Gin}{width=\maxwidth,height=\maxheight,keepaspectratio}
\setlength{\emergencystretch}{3em}  % prevent overfull lines
\providecommand{\tightlist}{%
  \setlength{\itemsep}{0pt}\setlength{\parskip}{0pt}}
\setcounter{secnumdepth}{5}
% Redefines (sub)paragraphs to behave more like sections
\ifx\paragraph\undefined\else
\let\oldparagraph\paragraph
\renewcommand{\paragraph}[1]{\oldparagraph{#1}\mbox{}}
\fi
\ifx\subparagraph\undefined\else
\let\oldsubparagraph\subparagraph
\renewcommand{\subparagraph}[1]{\oldsubparagraph{#1}\mbox{}}
\fi

% set default figure placement to htbp
\makeatletter
\def\fps@figure{htbp}
\makeatother


\title{Statistical Computing - Assessed Coursework 1}
\author{Conor Newton}
\date{}

\begin{document}
\maketitle

{
\setcounter{tocdepth}{3}
\tableofcontents
}
\newpage

\hypertarget{the-package}{%
\section{The Package}\label{the-package}}

I have created a package that has provides functions for both gradient
descent and stochastic gradient descent.

The package and all of the files for this project can be found
\href{https://www.github.com/conornewton/sc1-optimization}{here} on
github:

\url{https://www.github.com/conornewton/sc1-optimization}

\hypertarget{installation}{%
\subsection{Installation}\label{installation}}

This package can be installed directly from github using the following
command in an R shell if \texttt{devtools} is installed

\begin{Shaded}
\begin{Highlighting}[]
\NormalTok{    devtools}\OperatorTok{::}\KeywordTok{install_github}\NormalTok{(}\StringTok{"conornewton/sc1-optimization"}\NormalTok{)}
\end{Highlighting}
\end{Shaded}

This package has no required dependencies.

\hypertarget{documentation}{%
\subsection{Documentation}\label{documentation}}

The documentation for this package is generated automatically from the
source files using \texttt{roxygen2}.

The package exports two functions, \texttt{gradDescent} and
\texttt{stocGradDescent}. The documentation for them can be accessed
from the shell using the \texttt{?gradDescent} and
\texttt{?stocGradDescent} commands.

Alternatively, the documentation can be accessed in a pdf here:

\url{https://www.github.com/conornewton/sc1-optimization/doc/man.pdf}

\hypertarget{usage}{%
\subsection{Usage}\label{usage}}

Here are some basic example of how to use the \texttt{gradDescent} and
\texttt{stocGradDescent} functions.

Firstly, we can use gradDescent to estimate the argument of a local
minimum of the Rosenbrock function

\begin{Shaded}
\begin{Highlighting}[]
    \CommentTok{# Rosenbrock function}
\NormalTok{    f <-}\StringTok{ }\ControlFlowTok{function}\NormalTok{(x) (}\DecValTok{1} \OperatorTok{-}\StringTok{ }\NormalTok{x[}\DecValTok{1}\NormalTok{])}\OperatorTok{^}\DecValTok{2} \OperatorTok{+}\StringTok{ }\DecValTok{100} \OperatorTok{*}\StringTok{ }\NormalTok{(x[}\DecValTok{2}\NormalTok{] }\OperatorTok{-}\StringTok{ }\NormalTok{x[}\DecValTok{1}\NormalTok{]}\OperatorTok{^}\DecValTok{2}\NormalTok{)}\OperatorTok{^}\DecValTok{2}
    \KeywordTok{grad_descent}\NormalTok{(f, }\KeywordTok{c}\NormalTok{(}\DecValTok{0}\NormalTok{, }\DecValTok{0}\NormalTok{), }\DataTypeTok{n =} \DecValTok{100000}\NormalTok{, }\DataTypeTok{step_method =} \StringTok{"BB"}\NormalTok{)}
\end{Highlighting}
\end{Shaded}

Secondly, we can use stocGradDescent to estimate the argument that
minimises the mean square error

\begin{Shaded}
\begin{Highlighting}[]
    \CommentTok{# Summand of the objective function}
\NormalTok{    f <-}\StringTok{ }\ControlFlowTok{function}\NormalTok{(w, x, y) (}\KeywordTok{sum}\NormalTok{(w }\OperatorTok{*}\StringTok{ }\KeywordTok{c}\NormalTok{(}\DecValTok{1}\NormalTok{, x)) }\OperatorTok{-}\StringTok{ }\NormalTok{y)}\OperatorTok{^}\DecValTok{2}

    \CommentTok{# Generating a data set which will appear in the summand}
\NormalTok{    y <-}\StringTok{ }\KeywordTok{mapply}\NormalTok{(}\ControlFlowTok{function}\NormalTok{(x1, x2) }\KeywordTok{sum}\NormalTok{(}\KeywordTok{c}\NormalTok{(}\DecValTok{20}\NormalTok{, }\DecValTok{1}\NormalTok{) }\OperatorTok{*}\StringTok{ }\KeywordTok{c}\NormalTok{(x1, x2)) }\OperatorTok{+}\StringTok{ }\DecValTok{2}\NormalTok{, }\DecValTok{1}\OperatorTok{:}\DecValTok{100}\NormalTok{, }\DecValTok{1}\OperatorTok{:}\DecValTok{100}\NormalTok{)}
\NormalTok{    data <-}\StringTok{ }\KeywordTok{data.frame}\NormalTok{(}\DecValTok{1}\OperatorTok{:}\DecValTok{100}\NormalTok{, }\DecValTok{1}\OperatorTok{:}\DecValTok{100}\NormalTok{, y)}

    \KeywordTok{stoc_grad_descent}\NormalTok{(f, data, }\KeywordTok{c}\NormalTok{(}\DecValTok{0}\NormalTok{, }\DecValTok{0}\NormalTok{, }\DecValTok{0}\NormalTok{))}
\end{Highlighting}
\end{Shaded}

\hypertarget{testing}{%
\subsection{Testing}\label{testing}}

This package uses \texttt{testthat} for unit testing. Both the
\texttt{gradDescent} and \texttt{stocGradDescent} have unit tests to
ensure that they work for a variety of edge cases. These test are found
in the \texttt{tests/testthat} directory. If \texttt{testthat} is
installed locally, the package will be tested automatically on
installation.

After installation, the tests can be run manually using the following
command in the R shell

\begin{Shaded}
\begin{Highlighting}[]
\NormalTok{    devtools}\OperatorTok{::}\KeywordTok{test}\NormalTok{(}\StringTok{"sc1-optimization"}\NormalTok{)}
\end{Highlighting}
\end{Shaded}

\newpage

\hypertarget{gradient-descent}{%
\section{Gradient Descent}\label{gradient-descent}}

\newpage

\hypertarget{stochastic-gradient-descent}{%
\section{Stochastic Gradient
Descent}\label{stochastic-gradient-descent}}

\end{document}
