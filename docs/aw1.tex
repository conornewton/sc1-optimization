\PassOptionsToPackage{unicode=true}{hyperref} % options for packages loaded elsewhere
\PassOptionsToPackage{hyphens}{url}
%
\documentclass[12pt,]{article}
\usepackage{lmodern}
\usepackage{amssymb,amsmath}
\usepackage{ifxetex,ifluatex}
\usepackage{fixltx2e} % provides \textsubscript
\ifnum 0\ifxetex 1\fi\ifluatex 1\fi=0 % if pdftex
  \usepackage[T1]{fontenc}
  \usepackage[utf8]{inputenc}
  \usepackage{textcomp} % provides euro and other symbols
\else % if luatex or xelatex
  \usepackage{unicode-math}
  \defaultfontfeatures{Ligatures=TeX,Scale=MatchLowercase}
\fi
% use upquote if available, for straight quotes in verbatim environments
\IfFileExists{upquote.sty}{\usepackage{upquote}}{}
% use microtype if available
\IfFileExists{microtype.sty}{%
\usepackage[]{microtype}
\UseMicrotypeSet[protrusion]{basicmath} % disable protrusion for tt fonts
}{}
\IfFileExists{parskip.sty}{%
\usepackage{parskip}
}{% else
\setlength{\parindent}{0pt}
\setlength{\parskip}{6pt plus 2pt minus 1pt}
}
\usepackage{hyperref}
\hypersetup{
            pdftitle={Statistical Computing - Assessed Coursework 1},
            pdfauthor={Conor Newton},
            pdfborder={0 0 0},
            breaklinks=true}
\urlstyle{same}  % don't use monospace font for urls
\usepackage[margin=1in]{geometry}
\usepackage{color}
\usepackage{fancyvrb}
\newcommand{\VerbBar}{|}
\newcommand{\VERB}{\Verb[commandchars=\\\{\}]}
\DefineVerbatimEnvironment{Highlighting}{Verbatim}{commandchars=\\\{\}}
% Add ',fontsize=\small' for more characters per line
\usepackage{framed}
\definecolor{shadecolor}{RGB}{248,248,248}
\newenvironment{Shaded}{\begin{snugshade}}{\end{snugshade}}
\newcommand{\AlertTok}[1]{\textcolor[rgb]{0.94,0.16,0.16}{#1}}
\newcommand{\AnnotationTok}[1]{\textcolor[rgb]{0.56,0.35,0.01}{\textbf{\textit{#1}}}}
\newcommand{\AttributeTok}[1]{\textcolor[rgb]{0.77,0.63,0.00}{#1}}
\newcommand{\BaseNTok}[1]{\textcolor[rgb]{0.00,0.00,0.81}{#1}}
\newcommand{\BuiltInTok}[1]{#1}
\newcommand{\CharTok}[1]{\textcolor[rgb]{0.31,0.60,0.02}{#1}}
\newcommand{\CommentTok}[1]{\textcolor[rgb]{0.56,0.35,0.01}{\textit{#1}}}
\newcommand{\CommentVarTok}[1]{\textcolor[rgb]{0.56,0.35,0.01}{\textbf{\textit{#1}}}}
\newcommand{\ConstantTok}[1]{\textcolor[rgb]{0.00,0.00,0.00}{#1}}
\newcommand{\ControlFlowTok}[1]{\textcolor[rgb]{0.13,0.29,0.53}{\textbf{#1}}}
\newcommand{\DataTypeTok}[1]{\textcolor[rgb]{0.13,0.29,0.53}{#1}}
\newcommand{\DecValTok}[1]{\textcolor[rgb]{0.00,0.00,0.81}{#1}}
\newcommand{\DocumentationTok}[1]{\textcolor[rgb]{0.56,0.35,0.01}{\textbf{\textit{#1}}}}
\newcommand{\ErrorTok}[1]{\textcolor[rgb]{0.64,0.00,0.00}{\textbf{#1}}}
\newcommand{\ExtensionTok}[1]{#1}
\newcommand{\FloatTok}[1]{\textcolor[rgb]{0.00,0.00,0.81}{#1}}
\newcommand{\FunctionTok}[1]{\textcolor[rgb]{0.00,0.00,0.00}{#1}}
\newcommand{\ImportTok}[1]{#1}
\newcommand{\InformationTok}[1]{\textcolor[rgb]{0.56,0.35,0.01}{\textbf{\textit{#1}}}}
\newcommand{\KeywordTok}[1]{\textcolor[rgb]{0.13,0.29,0.53}{\textbf{#1}}}
\newcommand{\NormalTok}[1]{#1}
\newcommand{\OperatorTok}[1]{\textcolor[rgb]{0.81,0.36,0.00}{\textbf{#1}}}
\newcommand{\OtherTok}[1]{\textcolor[rgb]{0.56,0.35,0.01}{#1}}
\newcommand{\PreprocessorTok}[1]{\textcolor[rgb]{0.56,0.35,0.01}{\textit{#1}}}
\newcommand{\RegionMarkerTok}[1]{#1}
\newcommand{\SpecialCharTok}[1]{\textcolor[rgb]{0.00,0.00,0.00}{#1}}
\newcommand{\SpecialStringTok}[1]{\textcolor[rgb]{0.31,0.60,0.02}{#1}}
\newcommand{\StringTok}[1]{\textcolor[rgb]{0.31,0.60,0.02}{#1}}
\newcommand{\VariableTok}[1]{\textcolor[rgb]{0.00,0.00,0.00}{#1}}
\newcommand{\VerbatimStringTok}[1]{\textcolor[rgb]{0.31,0.60,0.02}{#1}}
\newcommand{\WarningTok}[1]{\textcolor[rgb]{0.56,0.35,0.01}{\textbf{\textit{#1}}}}
\usepackage{graphicx,grffile}
\makeatletter
\def\maxwidth{\ifdim\Gin@nat@width>\linewidth\linewidth\else\Gin@nat@width\fi}
\def\maxheight{\ifdim\Gin@nat@height>\textheight\textheight\else\Gin@nat@height\fi}
\makeatother
% Scale images if necessary, so that they will not overflow the page
% margins by default, and it is still possible to overwrite the defaults
% using explicit options in \includegraphics[width, height, ...]{}
\setkeys{Gin}{width=\maxwidth,height=\maxheight,keepaspectratio}
\setlength{\emergencystretch}{3em}  % prevent overfull lines
\providecommand{\tightlist}{%
  \setlength{\itemsep}{0pt}\setlength{\parskip}{0pt}}
\setcounter{secnumdepth}{5}
% Redefines (sub)paragraphs to behave more like sections
\ifx\paragraph\undefined\else
\let\oldparagraph\paragraph
\renewcommand{\paragraph}[1]{\oldparagraph{#1}\mbox{}}
\fi
\ifx\subparagraph\undefined\else
\let\oldsubparagraph\subparagraph
\renewcommand{\subparagraph}[1]{\oldsubparagraph{#1}\mbox{}}
\fi

% set default figure placement to htbp
\makeatletter
\def\fps@figure{htbp}
\makeatother


\title{Statistical Computing - Assessed Coursework 1}
\author{Conor Newton}
\date{}

\begin{document}
\maketitle

{
\setcounter{tocdepth}{3}
\tableofcontents
}
\newpage

\hypertarget{the-package}{%
\section{The Package}\label{the-package}}

I have created a package that provides functions for both gradient
descent and stochastic gradient descent.

The package and all of the files for this project can be found
\href{https://www.github.com/conornewton/sc1-optimization}{here} on
github:

\url{https://www.github.com/conornewton/sc1-optimization}

\hypertarget{installation}{%
\subsection{Installation}\label{installation}}

This package can be installed directly from github using the following
command in an R shell if \texttt{devtools} is installed

\begin{Shaded}
\begin{Highlighting}[]
\NormalTok{    devtools}\OperatorTok{::}\KeywordTok{install_github}\NormalTok{(}\StringTok{"conornewton/sc1-optimization"}\NormalTok{)}
\end{Highlighting}
\end{Shaded}

This package has no required dependencies.

\hypertarget{documentation}{%
\subsection{Documentation}\label{documentation}}

The documentation for this package is generated automatically from the
source files using \texttt{roxygen2}.

The package exports two functions, \texttt{grad\_descent} and
\texttt{stoc\_grad\_descent}. The documentation for them can be accessed
from the shell using the \texttt{?grad\_descent} and
\texttt{?stoc\_grad\_descent} commands.

Alternatively, the documentation can be accessed in a pdf here:

\url{https://www.github.com/conornewton/sc1-optimization/doc/man.pdf}

\hypertarget{testing}{%
\subsection{Testing}\label{testing}}

This package uses \texttt{testthat} for unit testing. Both the
\texttt{gradDescent} and \texttt{stocGradDescent} have unit tests to
ensure that they work for a variety of edge cases. These test are found
in the \texttt{tests/testthat} directory. If \texttt{testthat} is
installed locally, the package will be tested automatically on
installation.

After installation, the tests can be run manually using the following
command in the R shell

\begin{Shaded}
\begin{Highlighting}[]
\NormalTok{    devtools}\OperatorTok{::}\KeywordTok{test}\NormalTok{(}\StringTok{"sc1-optimization"}\NormalTok{)}
\end{Highlighting}
\end{Shaded}

\newpage

\hypertarget{gradient-descent}{%
\section{Gradient Descent}\label{gradient-descent}}

This focuses on the methodology of my gradient descent implementation.
The relevant source code and tests can be found at the following urls

\begin{itemize}
\tightlist
\item
  \url{https://github.com/conornewton/sc1-optimization/blob/master/R/gradDescent.R}
\item
  \url{https://github.com/conornewton/sc1-optimization/blob/master/tests/testthat/testGradDescent.R}
\end{itemize}

\hypertarget{introduction}{%
\subsection{Introduction}\label{introduction}}

Gradient descent methods attempt to find a local minimum of a
differentiable functions \(f\) by moving along the curve in the opposite
direction to the gradient at each point. Since the gradient gives the
direction of the ``fastest increase'' moving in the opposite direction
should gives us the fastest decrease. Gradient descent is performed by
iterating the following \[
    \mathbf{x}_n = \mathbf{x}_{n-1} - \gamma_n \nabla f(\mathbf{x}_{n-1})
\]

Here \(\gamma_n\) is the step-size. This determines how much we should
move in the opposite direction of gradient in each step. For simplicity,
it can be choose to be constant. More effective gradient descent
approaches use a systematic way of determining \(\gamma_n\) on each
iteration, such as a line search.

\hypertarget{the-gradient}{%
\subsection{The Gradient}\label{the-gradient}}

For our gradient descent implementation, we have the option of passing
as a parameter the gradient of the function we would like to minimize
otherwise it will estimate the gradient at each point using finite
differencing.

Finite differencing at a point \(\mathbf{x}\) works by finding the
gradient of the line passing through two points on each side and close
to \(\mathbf{x}\). Each component of the gradient is computed by the
following \[
(\nabla f (\mathbf{x}))^{(i)} \approx \frac{f(\mathbf{x} + \Delta \mathbf{e}_i) - f(\mathbf{x} - \Delta \mathbf{e}_i)}{2\Delta}
\] where \(\Delta\) is a small constant.

The gradient is also used to decide when we have found a local minimum.
If the magnitude of the gradient is small, then we are a point with
little curvature and therefore close to a local minimum. For our
gradient descent implementation, the user can provide a tolerance for
the magnitude of the gradient to decide when to terminate the search.

\hypertarget{step-sizes}{%
\subsection{Step Sizes}\label{step-sizes}}

In this package, we have the choice of using the following three
different approaches to calculate step size

\begin{itemize}
\tightlist
\item
  Constant step size
\item
  Line Search
\item
  Barzilai-Borwein
\end{itemize}

All three of these approaches share the same guarantee that gradient
descent will converge to a local minimum if \(f\) is convex and its
gradient \(\nabla f\) is Lipschitz.

\hypertarget{constant-step-size}{%
\subsubsection{Constant Step Size}\label{constant-step-size}}

Choosing a constant step size is the most basic approach to gradient
descent. For every iteration, our estimate for the minimum
\(\mathbf{x}\) moves by the same distance in the opposite direction of
the gradient. This can often lead to overshooting/undershooting the
minimum in that direction, therefore this approach can take long to
converge. The next two methods both give faster convergence.

\hypertarget{backtracking-line-search}{%
\subsubsection{Backtracking Line
Search}\label{backtracking-line-search}}

Line search selects the step size \(\gamma_n\) that satisfies the
following
\[ \gamma_n = \underset{\gamma}{\text{argmin}} f(\mathbf{x}_{n-1} - \gamma\nabla f(\mathbf{x}))\]
This is finding finding the argument \(\gamma\) that minimises \(f\) on
the line \((\mathbf{x}_{n-1} - \gamma\nabla f(\mathbf{x}))\).

A backtracking approach can be used to quickly estimate this line search
step size. This works by iterating \[ \alpha_n = \tau \alpha_{n - 1}\]
where \(\alpha_n\) is the step size and \(\tau \in (0, 1)\) until the
following is satisfied
\[{f(\mathbf{x}) - f(\mathbf{x} - \alpha_j \nabla f(\mathbf{x}))} \ge c \alpha_j ||\nabla f(\mathbf{x})||^2\]
for some \(c \in (0, 1)\). Essentially we are just looking for a
sufficient drop in the objective function in the direction of
\(\nabla f(\mathbf{x})\).

\hypertarget{barzilai-borwein-method}{%
\subsubsection{Barzilai-Borwein Method}\label{barzilai-borwein-method}}

A third approach to calculating the step size is the Barzilai-Borwein
method. This approach takes into consideration the previous estimate
\(\mathbf{x}_{n-1}\). The step size is calculated as follows \[
    \gamma_n = \frac{(\mathbf{x} - \mathbf{x}_{n-1}) \cdot (\nabla f(\mathbf{x}_n) - \nabla f(\mathbf{x}_{n-1}))}{||\nabla f(\mathbf{x}_n) - \nabla f(\mathbf{x}_{n-1})||^2}
\]

\hypertarget{usage}{%
\subsection{Usage}\label{usage}}

Here we give some examples of how we can use the \texttt{grad\_descent}
function.

Firstly, we can use \texttt{grad\_descent} to estimate the argument of
the local minimum of the Rosenbrock function using a constant step size.
Our initial guess is the point \((0,0)\) and our optimization will not
exceed 100000 iterations.

\begin{Shaded}
\begin{Highlighting}[]
    \CommentTok{# Rosenbrock function}
\NormalTok{    f <-}\StringTok{ }\ControlFlowTok{function}\NormalTok{(x) (}\DecValTok{1} \OperatorTok{-}\StringTok{ }\NormalTok{x[}\DecValTok{1}\NormalTok{])}\OperatorTok{^}\DecValTok{2} \OperatorTok{+}\StringTok{ }\DecValTok{100} \OperatorTok{*}\StringTok{ }\NormalTok{(x[}\DecValTok{2}\NormalTok{] }\OperatorTok{-}\StringTok{ }\NormalTok{x[}\DecValTok{1}\NormalTok{]}\OperatorTok{^}\DecValTok{2}\NormalTok{)}\OperatorTok{^}\DecValTok{2}
    \KeywordTok{grad_descent}\NormalTok{(f, }\KeywordTok{c}\NormalTok{(}\DecValTok{0}\NormalTok{, }\DecValTok{0}\NormalTok{), }\DataTypeTok{n =} \DecValTok{100000}\NormalTok{, }\DataTypeTok{step_method =} \FloatTok{0.01}\NormalTok{)}
\end{Highlighting}
\end{Shaded}

This can be estimated much quicker if we choose a different step
methods.

\begin{Shaded}
\begin{Highlighting}[]
    \CommentTok{# Uses Barzilai-Borwein}
\NormalTok{    f <-}\StringTok{ }\ControlFlowTok{function}\NormalTok{(x) (}\DecValTok{1} \OperatorTok{-}\StringTok{ }\NormalTok{x[}\DecValTok{1}\NormalTok{])}\OperatorTok{^}\DecValTok{2} \OperatorTok{+}\StringTok{ }\DecValTok{100} \OperatorTok{*}\StringTok{ }\NormalTok{(x[}\DecValTok{2}\NormalTok{] }\OperatorTok{-}\StringTok{ }\NormalTok{x[}\DecValTok{1}\NormalTok{]}\OperatorTok{^}\DecValTok{2}\NormalTok{)}\OperatorTok{^}\DecValTok{2}
    \KeywordTok{grad_descent}\NormalTok{(f, }\KeywordTok{c}\NormalTok{(}\DecValTok{0}\NormalTok{, }\DecValTok{0}\NormalTok{), }\DataTypeTok{n =} \DecValTok{100000}\NormalTok{, }\DataTypeTok{step_method =} \StringTok{"BB"}\NormalTok{)}
\end{Highlighting}
\end{Shaded}

\begin{Shaded}
\begin{Highlighting}[]
    \CommentTok{# Uses Backtracking Line Search}
\NormalTok{    f <-}\StringTok{ }\ControlFlowTok{function}\NormalTok{(x) (}\DecValTok{1} \OperatorTok{-}\StringTok{ }\NormalTok{x[}\DecValTok{1}\NormalTok{])}\OperatorTok{^}\DecValTok{2} \OperatorTok{+}\StringTok{ }\DecValTok{100} \OperatorTok{*}\StringTok{ }\NormalTok{(x[}\DecValTok{2}\NormalTok{] }\OperatorTok{-}\StringTok{ }\NormalTok{x[}\DecValTok{1}\NormalTok{]}\OperatorTok{^}\DecValTok{2}\NormalTok{)}\OperatorTok{^}\DecValTok{2}
    \KeywordTok{grad_descent}\NormalTok{(f, }\KeywordTok{c}\NormalTok{(}\DecValTok{0}\NormalTok{, }\DecValTok{0}\NormalTok{), }\DataTypeTok{n =} \DecValTok{100000}\NormalTok{, }\DataTypeTok{step_method =} \StringTok{"BLS"}\NormalTok{)}
\end{Highlighting}
\end{Shaded}

We can give our \texttt{grad\_descent} function an explicit gradient of
\(f\) and choose a tolerance for our gradient as follows

\begin{Shaded}
\begin{Highlighting}[]
\NormalTok{    f <-}\StringTok{ }\ControlFlowTok{function}\NormalTok{(x) x[}\DecValTok{1}\NormalTok{]}\OperatorTok{^}\DecValTok{2} \OperatorTok{+}\StringTok{ }\NormalTok{x[}\DecValTok{2}\NormalTok{]}\OperatorTok{^}\DecValTok{2} \OperatorTok{+}\StringTok{ }\DecValTok{3} \OperatorTok{*}\StringTok{ }\NormalTok{x[}\DecValTok{1}\NormalTok{] }\OperatorTok{+}\StringTok{ }\DecValTok{2} \OperatorTok{*}\StringTok{ }\NormalTok{x[}\DecValTok{2}\NormalTok{] }\OperatorTok{+}\StringTok{ }\DecValTok{9}
\NormalTok{    grad_f <-}\StringTok{ }\ControlFlowTok{function}\NormalTok{(x) }\KeywordTok{c}\NormalTok{(}\DecValTok{2} \OperatorTok{*}\StringTok{ }\NormalTok{x[}\DecValTok{1}\NormalTok{] }\OperatorTok{+}\StringTok{ }\DecValTok{3}\NormalTok{, }\DecValTok{2} \OperatorTok{*}\StringTok{ }\NormalTok{x[}\DecValTok{2}\NormalTok{] }\OperatorTok{+}\StringTok{ }\DecValTok{2}\NormalTok{)}
    \KeywordTok{grad_descent}\NormalTok{(f, }\KeywordTok{c}\NormalTok{(}\DecValTok{0}\NormalTok{, }\DecValTok{1}\NormalTok{), grad_f, }\DataTypeTok{tol =} \FloatTok{1e-4}\NormalTok{)}
\end{Highlighting}
\end{Shaded}

This summarises pretty much all of the functionality of the
\texttt{grad\_descent} function. \newpage

\hypertarget{stochastic-gradient-descent}{%
\section{Stochastic Gradient
Descent}\label{stochastic-gradient-descent}}

This focuses on the methodology of my gradient descent implementation.
The relevant source code and tests can be found at the following urls

\begin{itemize}
\tightlist
\item
  \url{https://github.com/conornewton/sc1-optimization/blob/master/R/gradDescent.R}
\item
  \url{https://github.com/conornewton/sc1-optimization/blob/master/tests/testthat/testGradDescent.R}
\end{itemize}

\hypertarget{introduction-1}{%
\subsection{Introduction}\label{introduction-1}}

Stochastic gradient descent is a gradient descent method that is
commonly used to find minimum of a function that depends on a large data
set.

Instead, if we can break up the objective function into \(n\) pieces \[
Q(\mathbf{x}) = \sum_{i=1}^nQ_i(\mathbf{x})
\] where each \(Q_i\) only involves using the i'th data point, we can
then apply stochastic gradient descent. Functions of this form are
common in machine learning, such as mean square error, log-likelihood
etc.

Stochastic gradient descent can be performed by iterating the following
\[
    \mathbf{x}_{n} = \mathbf{x}_{n-1} - \gamma \nabla Q_i(\mathbf{x})
\] In each iteration we are only considering a single data point, but by
the end of the computation all of the data points have been involved in
learning \(\mathbf{x}\). For our simple implementation we have set
\(\gamma\) to be constant.

\hypertarget{usage-1}{%
\subsection{Usage}\label{usage-1}}

We can use stoc\_grad\_descent to estimate the argument that minimises
the mean square error as follows

\begin{Shaded}
\begin{Highlighting}[]
    \CommentTok{# Summand of the objective function}
\NormalTok{    f <-}\StringTok{ }\ControlFlowTok{function}\NormalTok{(w, x, y) (}\KeywordTok{sum}\NormalTok{(w }\OperatorTok{*}\StringTok{ }\KeywordTok{c}\NormalTok{(}\DecValTok{1}\NormalTok{, x)) }\OperatorTok{-}\StringTok{ }\NormalTok{y)}\OperatorTok{^}\DecValTok{2}

    \CommentTok{# Generating a data set which will appear in the summand}
\NormalTok{    y <-}\StringTok{ }\KeywordTok{mapply}\NormalTok{(}\ControlFlowTok{function}\NormalTok{(x1, x2) }\KeywordTok{sum}\NormalTok{(}\KeywordTok{c}\NormalTok{(}\DecValTok{20}\NormalTok{, }\DecValTok{1}\NormalTok{) }\OperatorTok{*}\StringTok{ }\KeywordTok{c}\NormalTok{(x1, x2)) }\OperatorTok{+}\StringTok{ }\DecValTok{2}\NormalTok{, }\DecValTok{1}\OperatorTok{:}\DecValTok{100}\NormalTok{, }\DecValTok{1}\OperatorTok{:}\DecValTok{100}\NormalTok{)}
\NormalTok{    data <-}\StringTok{ }\KeywordTok{data.frame}\NormalTok{(}\DecValTok{1}\OperatorTok{:}\DecValTok{100}\NormalTok{, }\DecValTok{1}\OperatorTok{:}\DecValTok{100}\NormalTok{, y)}

    \KeywordTok{stoc_grad_descent}\NormalTok{(f, data, }\KeywordTok{c}\NormalTok{(}\DecValTok{0}\NormalTok{, }\DecValTok{0}\NormalTok{, }\DecValTok{0}\NormalTok{))}
\end{Highlighting}
\end{Shaded}

\end{document}
